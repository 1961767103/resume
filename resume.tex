% !TEX TS-program = xelatex
% !TEX encoding = UTF-8 Unicode
% !Mode:: "TeX:UTF-8"

\documentclass{resume}
\usepackage{zh_CN-Adobefonts_external} % Simplified Chinese Support using external fonts (./fonts/zh_CN-Adobe/)
%\usepackage{zh_CN-Adobefonts_internal} % Simplified Chinese Support using system fonts
\usepackage{linespacing_fix} % disable extra space before next section
\usepackage{cite}
\usepackage{geometry} % 添加 geometry 宏包
\usepackage{fontsize} % 添加 fontsize 宏包
\usepackage{setspace} % 添加 setspace 宏包

\changefontsize{10pt} % 11pt 是字体大小
% 重新定义 \datedsubsection 命令,设置字体大小为 12pt
\renewcommand{\datedsubsection}[2]{
  \subsection*{\fontsize{10pt}{14pt}\selectfont \textbf{#1} \hfill #2}
}

% 设置页面边距
\geometry{
  a4paper, % 纸张大小
  left=0.5cm, % 左边距
  right=0.5cm, % 右边距
  top=0.5cm, % 上边距
  bottom=0.5cm, % 下边距
}
\linespread{1.05}

\begin{document}
\pagenumbering{gobble} % suppress displaying page number

\name{吴泽森}

\basicInfo{
  \email {}wkw4399xs@gmail.com  \textperiodcentered\ 
  \phone{13642204176} \textperiodcentered\ 
  \github []{}LinQi0777
}
 
\section{\faGraduationCap\  教育经历}
\datedsubsection{\textbf{广东石油化工学院(本科)}  \quad   \quad   \quad \quad \quad GPA:3.78 / 4.0 \quad \quad \quad \quad \textit{数据科学与大数据技术} \quad \quad   }{2021.09 -- 2025.07}
 \begin{itemize}
  \item \textit{Apache、Opengoofy 开源社区贡献者,Apache ShardingSphere、type-fest(14k+ star ) Contributor、AWS 生成式 AI 社区成员}。
  \item \textit{广东省重点实验室主力开发,负责与三桶油、中科炼化等石化企业项目对接,并独立完成了中科炼化石化管家项目的研发。}
  \item \textit{热爱分享,知识星球粉丝 600+,多次排查并解决过生产问题,如 百万级消息堆积 、内存溢出、GC 毛刺等,具备 SQL 优化经验。}
  \item  \textit{参与过支付、风控、大模型相关业务的开发工作并具有项目开发经验,Github 开源优惠券防薅羊毛风控组件收获 star 150+。 }
\end{itemize}

\section{\faCogs\ 专业技能}
% increase linespacing [parsep=0.5ex]
\begin{itemize}[parsep=0.5ex]
  \item \textbf{操作系统}:熟悉操作系统相关理论,如内存管理、进程通信等,熟悉        Select、Poll、Epoll        等模型的原理。
  \item \textbf{计算机网络}:熟悉计算机网络基础知识,如        TCP、UDP、QUIC、HTTPS        等网络协议、流量控制、拥塞控制、DNS        等。
  \item \textbf{MySQL}:熟悉        MySQL        使用及原理,如索引、事务、锁、日志等,具有数据库以及        SQL        调优经验。
    \item \textbf{Redis}:熟悉        Redis        核心数据结构、数据持久化、数据过期、内存淘汰等设计,深入理解        Redis        高可用架构。
    \item \textbf{JVM}:熟悉        JVM        底层原理,如类加载、垃圾回收算法等,熟悉        JVM        垃圾回收器的使用以及核心参数,实习过程中有过        JVM        线上调优经验,解决过生产应用        CPU        100\%\ 、内存溢出、GC        回收毛刺等线上问题。
    \item \textbf{多线程并发}:熟悉        JUC        并发工具类的使用和原理,如线程池、锁、阻塞队列以及原子并发类等。
\item \textbf{Spring 全家桶}:熟练使用        SpringBoot、Spring        Cloud        Alibaba        进行        Web        后端开发,熟悉部分原理,如        IOC、AOP。        
\item \textbf{Golang}:熟悉        Map、Slice、Channel        等底层实现,熟悉        GMP        模型、内存分配等机制,熟悉        Gin、Gorm、Hertz        等框架使用。
\item \textbf{消息队列}:熟悉        RocketMQ、Kafka        消息中间件的使用,实际项目中解决过重消费、消息积压问题。
\item \textbf{分布式}:熟悉分布式、微服务等架构,熟悉分布式锁、分布式 ID 等,了解过        Paxos、Raft、Gossip、一致性哈希等理论。
\item \textbf{分布式数据库}:熟悉        HBase、MongoDB、ClickHouse        的使用,并了解部分原理,如一致性原理、        LSM        树、WiredTiger        等。
\end{itemize}


\section{\faUsers\ 实习经历}
\datedsubsection{\hspace{5em}\textbf{货拉拉} }{2024.10 -- 至今}
\role{大数据产品与技术部-基础架构组-基础架构开发实习生|SpringBoot、Hive、Spark、Flink、Dubbo、Kafka}

\begin{itemize}
  \item 基于布隆过滤器完成了春节拉货节期间\textbf{亿级数量重复请求}的过滤,\textbf{降低了 ES  97\%\ 的数据库压力},利用更便宜的 Redis 集群代替 ES 集群,实现降本增效,降低了约 52 \%\ 的成本,并对 ES 进行配置与参数优化, 移除 SLB 依赖,减少 99 \%\ 的 YoungGC 次数 。


  \item 基于 SPI 机制扩展了        RPC        框架        CGLib、ASM        等动态代理和反射调用,并实现了一致性哈希、轮询等增强型负载均衡策略。

  \item {独立排查并解决线上问题:\textbf{JIT逆优化}导致 CPU 100\%\ ,\textbf{元空间频繁 FullGC},\textbf{内存泄漏导致无限        CMS        GC}。}
  \item 独立完成 Hive 回归测试方案的制定以及测试混合引擎的开发,并使用 Disruptor 替代 BlockingQueue 实现生产—消费模型解耦。

\end{itemize}

\datedsubsection{\hspace{2em}\textbf{莉莉丝游戏}}{2024.04 -- 2024.10}
\role{远光 84 游戏开发部-服务器组-服务器开发工程师(实习)|Golang、C++、Gin、MongoDB、Redis、gRPC、Dify}{}
\begin{itemize}
  \item 主要产出:        \href{https://jcn9lm5pjtx7.feishu.cn/wiki/AeyhwuVfIiQuO7kEysTcLhfUn4d?from=from_copylink}{\textbf{莉莉丝游戏通用 CDN 最佳实践}}、\href{https://jcn9lm5pjtx7.feishu.cn/wiki/ALS2whLI0iGgQskPk30cdr1QnJf}{\textbf{C++编译时间优化}},大模型监控-评审 SDK 入选飞书先锋        AI        大模型决赛。

  \item 独立完成        CDN        质量探测工具的开发,运用了\textbf{零拷贝}、\textbf{对象池}等思想提高下载速率,使用\textbf{滑动窗口算法}进行下载速率统计,并完成了\textbf{路由追踪}、\textbf{二次回源检测}等功能,最终得出 \textbf{CDN 调度域}的结论,开启调度域后\textbf{平均下载速率提高}        50\%\ ,成本降低 30\%\ 。

  \item 使用        pprof        对        Go        服务进行性能分析,排查并定位        \textbf{gRPC        协程池溢出}、\textbf{融合日志服务内存泄漏}等问题,并独立进行解决。
  \item 优化        SingleFlight,使其支持设置\textbf{超时、数据版本控制、指数退避以及均匀重试、分布式锁},作为解决\textbf{缓存击穿}通用组件。
  \item 基于 Redis Hash 和 Zset 结构封装 Go 版本的 Tairzset 实现多维度排行榜功能,实现降本增效,并结合时间轮解决大Key问题。
  \item 参与莉莉丝游戏测试平台配置中心的开发,完成项目启动初始化配置加载功能与配置动态变更功能的实现。
\end{itemize}



% Reference Test
%\datedsubsection{\textbf{Paper Title\cite{zaharia2012resilient}}}{May. 2015}
%An xxx optimized for xxx\cite{verma2015large}
%\begin{itemize}
%  \item main contribution
%\end{itemize}












\section{\faCode 项目经历}

\datedsubsection{\textbf{DaoBox——基于区块链技术的 NFT 数字藏品交易平台\&\ 可支持日活量 3 W+ \&\ 单日交易量最高 15W}}{}
\setlength{\parindent}{0pt}
\textit{\textbf{项目描述}:}{基于 Hyperchain 联盟链研发的 NFT 数字藏品平台,支持藏品的铸造、交易等核心功能,并提供空投、转增等核心玩法。} \\
\textit{\textbf{技术栈}:SpringBoot、Spring        Cloud        Alibaba、MySQL、Redis、RocketMQ、ElasticSearch、Seata、RocksDB、Serverless
}{}

\begin{itemize}
    \item 独立完成用户\textbf{订单超时}的数据处理,并完成订单数据的\textbf{冷热分离}实现,提高数据库读写性能。
    \item 排查并解决了信息同步系统中        RocketMQ        的百万级别数据量的\textbf{消息堆积}以及消息\textbf{重消费问题}。
    \item 重写 ShardingSphere 和 SnakeYaml 源码,解决 Seata 集成和兼容问题;修改 Canal 源码,解决藏品同步过程时间转换异常问题。
    \item 基于 Redis 预扣减+MQ 异步化+MySQL乐观锁,实现\textbf{TPS 700}的藏品秒杀抢购,并解决库存超卖问题。
    \item 参与系统网关模块的开发,通过对称与非对称加密算法实现网关数字签名与 URL 动态加密机制。
    \item 采用 HotKey 热点探测+二级缓存预热,实时识别 Top20 热点藏品并预热至边缘节点,详情页加载速度从 \textbf{1.2s 优化至200ms}。
    \item 独立完成项目支付模块的开发工作,实现多渠道支付对接,并在开发过程中解决了支付幂等以及 Seata 事务失效问题。

\end{itemize}


\end{document}
